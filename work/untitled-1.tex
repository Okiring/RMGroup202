\documentclass[12pt,letterpaper]{article}
\begin{document}

\title{COLLEGE OF COMPUTING AND INFORMATION SCIENCES\\ DEPARTMENT OF COMPUTER SCIENCE\\ SCHOOL OF COMPUTING AND INFORMATICS TECHNOLOGY\\}
\maketitle
\title{NAME:	     OKIRING PAUL\\
STUDENTS NUMBER:	 214016765\\
REGISTRATION NUMBER: 14/U/13973/EVE}
\maketitle

\section{INTRODUCTION}

     Model checking has emerged as a powerful method for the formal verification of programs. Temporal logics such as CTL (computational tree logic) and CTL* are widely used to specify programs because they are expressive and easy to understand. Given an abstract model of a program, a model checker (which typically implements the acceptance problem for a class of automata) verifies whether the model meets a given specification. A conceptually attractive method for solving the model checking problem is by reducing it to the solution of (a suitable subclass of) parity games. These are a type of two player infinite game played on a finite graph.
\section{PROBLEM STATEMENT}
Given a model of a system, exhaustively and automatically check whether this model meets a given specification. Typically, one has hardware or software systems in mind, whereas the specification contains safety requirements such as the absence of deadlocks and similar critical states that can cause the system to crash. Model checking is a technique for automatically verifying correctness properties of finite-state systems.
In order to solve such a problem algorithmically, both the model of the system and the specification are formulated in some precise mathematical language. To this end, the problem is formulated as a task in logic, namely to check whether a given structure satisfies a given logical formula. This general concept applies to many kinds of logics and suitable structures. A simple model checking problem is verifying whether a given formula in the propositional logic is satisfied by a given structure.

\section{BACKGROUND}
Model checking did not arise in a historical vacuum. There was an important problem that needed to be solved, namely concurrent problem verification. Concurrency errors are particularly difficult to find by program testing, since they are often hard to reproduce. Most of the formal research on this topic involved constructing proofs by hand using Floyd-Hoare style logic. Probably, the best known formal system was the one proposed by Owicki and Gries for reasoning about conditional critical regions.
Also, in the late 1970’s, Pnueli and Owicki and Lambert had proposed the use of temporal logic for specifying concurrent programs. Although they still advocated hand constructed proofs, their work demonstrated convincingly that temporal logic was ideal for expressing concepts mutual exclusion, absence of deadlock, and absence of starvation.

\section{MAIN OBJECTIVES}
 To show the connexions between the temporal logics CTL and / or CTL*, automata, and games.
\subsection{OBJECTIVES}
  \begin{enumerate}
    \item Representing CTL / CTL* as classes of alternating tree automata\\
    \item Inter-translation between CTL / CTL* and classes of alternating tree automata\\
    \item Using B¨uchi games and other subclasses of parity games to analyze the CTL / CTL* model checking problem\\
    \item Efficient implementation of model checking algorithms\\
    \item Application of the model checker to higher-order model checking\\
   
\section{ Methodology}

The decision procedures for these logics are known;
\paragraph{ The tree automatas}
	a tree automaton is a type of state machine that deals with tree structures,rather than the strings of more conventional state machines.
\paragraph{ Tableaux}
	this is a decision method for propositional temporal logic that can be used to formalize reasoning about time and temporal relations.
\paragraph{Axiom systems}
	An axiomatic system is any set of axioms from which some or all axioms can be used in conjuction to logically derive theorems.
Due to the ability of temporal logic to allow us to make deductive arguments about not only what is,but what was , what will be and what will always be, this has led to the application of the logics to specifications and verification of reactive and concurrent programs and systems i.e games.
The key temporal patterns of importance in specifying such programs are:
\subparagraph{“Liviness” properties or eventualities }– which ensure a specific precondition is intially satisfied. Then a desiarable state is eventually reached.
\subparagraph{“Safety” or “Invariance” properties} – this ensures that a specific precondition is intially satisfied, then undesirable state will never occur.
\subparagraph{“Fairness” properties} -Fairness requires that in a system where several processes sharing resources are run concurrently , they must be treated fairly by the program.
\paragraph{Artificial Intelligence}
	Temporal reasoning has also been naturally combined with other well developed framework for AI such as the situation calculas(Pinto and Rieter 1995) and action theory ( Lamport 1994)

Traditionally methods such as temporal arguments in which the temporaldimensions is captured by augmenting each time variable proposition or predicate with an extra argument place to be filled by an expression designating a time.

  \end{enumerate}

\end{document} 
