\documentclass[10pt,a4paper]{report}
\usepackage[utf8]{inputenc}
\usepackage{amsmath}
\usepackage{amsfonts}
\usepackage{amssymb}
\begin{document}
\section{ Methodology}

The decision procedures for these logics are known;
\paragraph{ The tree automatas}
	a tree automaton is a type of state machine that deals with tree structures,rather than the strings of more conventional state machines.
\paragraph{ Tableaux}
	this is a decision method for propositional temporal logic that can be used to formalize reasoning about time and temporal relations.
\paragraph{Axiom systems}
	An axiomatic system is any set of axioms from which some or all axioms can be used in conjuction to logically derive theorems.
Due to the ability of temporal logic to allow us to make deductive arguments about not only what is,but what was , what will be and what will always be, this has led to the application of the logics to specifications and verification of reactive and concurrent programs and systems i.e games.
The key temporal patterns of importance in specifying such programs are:
\subparagraph{“Liviness” properties or eventualities }– which ensure a specific precondition is intially satisfied. Then a desiarable state is eventually reached.
\subparagraph{“Safety” or “Invariance” properties} – this ensures that a specific precondition is intially satisfied, then undesirable state will never occur.
\subparagraph{“Fairness” properties} -Fairness requires that in a system where several processes sharing resources are run concurrently , they must be treated fairly by the program.
\paragraph{Artificial Intelligence}
	Temporal reasoning has also been naturally combined with other well developed framework for AI such as the situation calculas(Pinto and Rieter 1995) and action theory ( Lamport 1994)

Traditionally methods such as temporal arguments in which the temporaldimensions is captured by augmenting each time variable proposition or predicate with an extra argument place to be filled by an expression designating a time.
\end{document}